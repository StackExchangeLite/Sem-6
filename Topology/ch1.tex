\documentclass[main.tex]{subfiles}

\begin{document}
\chapter{Topological Spaces and Continuous functions}
\section{Topological Spaces}
\subsection{Basic notions, bases, subbases etc.}
\defn{Topology}{A \emph{topology} $\tau$ on a set $X$ is a collection $\mathscr{B}$ of sets called \emph{open sets} such that:
\begin{enumerate}
\item $\varnothing$ and $X$ belong to $\mathscr{B}$
\item For any collection of sets $U_{\alpha} \in \mathscr{B}$, $\cup_{\alpha} U_{\alpha}$ is also in $\mathscr{B}$ (closed under arbitrary unions)
\item For any finite collection $\{U_1,U_2 \cdots U_k\}$ of sets of $\mathscr{B}$, $\cap_{i=1}^{k}U_i \in \mathscr{B}$ (closed under finite intersections)
\end{enumerate}}
\defn{Discrete and Indiscrete topology}{\begin{enumerate}
    \item The topology $(\varnothing,X)$ of a set $X$ is called the \emph{indiscrete topology}.
    \item The topology in which every subset of $X$ is an open set is called the \emph{Discrete topology}
\end{enumerate} }
\defn{Finer and Coarser topologies}{Let $\tau$ and $\tau'$ be two topologies of space $X$. We say $\tau'$ is \emph{finer} than $\tau$ if $\tau \subset \tau'$, or every open set in $\tau$ is one in $\tau'$. In other words, $\tau$ is \emph{coarser} than $\tau$.}
\begin{remark}
    Of course, not all topologies are comparable, like the above definition suggests. For example, consider the 3 element space $\{x,y,z\}$. $\{\varnothing,(x),(x,y),(x,y,z)\}$ is a topology as can easily be seen. Likewise, $\{\varnothing,(x),(y,z),(x,y,z)\}$ is also a topology, but these two are not comparable, since their symmetric difference is non-empty.
\end{remark}
\begin{example}
    \textbf{The finite complement topology:} Let $X$ be a set. Consider the set of all $U \in X$ such that $U^C$ is either finite or $X$. Obviously, $\varnothing$ is in this collection, as is $X$. Let $\{U_{\alpha}:\alpha \in A\}$ by an arbitrary collection of such sets. Consider $(\cup_{\alpha} U_{\alpha})^C=\cap_{\alpha} U^{C}_{\alpha}$ which is easily seen to be finite. Hence, closed under arbitrary unions.
    Consider $\{U_1,U_2 \cdots U_k\}$ of such sets. $(\cap_{i=1}^{k}U_{i})^C=\cup_{i=1}^{n} U^{C}_i$ which is, at max, finite. Hence, closed under finite intersections. Therefore, this collection is a topology and is called the \emph{Finite Complement Topology}.
\end{example}

\defn{Basis for a topology}{Let $X$ be a space. $C$ is called a \emph{basis} for $X$ if the following are satisfied:
\begin{enumerate}
    \item For every $x \in X$, there exists a $C_x$ in $C$ so that $x \in C_x$
    \item If $x$, $C_1$ and $C_2$ are such that $x \in C_1 \cap C_2 \neq \varnothing$, then there exists $C_{12} \in C$ such that $x \in C_{12} \subseteq C_{1} \cap C_{2}$ 
\end{enumerate}}

\defn{Topology \emph{generated} by a basis $\C$ of $X$}{The topology $\tau_{\C}$ generated by the basis $\C$ is defined as the collection $\U$ of subsets of $X$ such that if $U \in \U$, and $x \in U$, then there exists $C \in \C$ such that $x \in C \subseteq U$. It is obvious that $\C$ itself is contained in $\U$. }
\thmp{The topology $\tau_{\C}$ generated by $\C$, a basis, does indeed form a topology.}{Does $X$ belong to $\tau_{\C}$? let $x \in X$. By virtue of being a basis, there exists a $C_x \subseteq X$ such that $x \in C_x$. Hence, $X \in \tau_{\C}$. Does $\varnothing \in \tau_{\C}$? Vacuously so. Let $U_{\alpha}$ be open in $\tau_{\C}$. Consider $\cup_{\alpha} U_{\alpha}$ and $x \in \cup_{\alpha} U_{\alpha}$. This means $x \in U_{\delta}$ for some $\delta$. That means that there exists $C_{x}$ so that $x \in C_x \subseteq U_{\delta} \subseteq \cup_{\alpha} U_{\alpha}$. This means $\tau_{\C}$ is closed under arbitrary union.
Consider singleton collections, for the moment, of the kind $\{U_1\}$. It is obvious that the intersection of every element in this collection is part of $\tau_{\C}$. Consider a two sized collection $\{U_1,U_2\}$. Let $x \in U_1 \cap U_2$. Obviously, $x \in C_1 \subseteq U_1$ and $x \in C_2 \subseteq U_2$. From the definition of basis, there exists a basis element $C$ so that $x \in C \subseteq C_1 \cap C_2 \subseteq U_1 \cap U_2$ making $\cap_{i=1}^2 U_i$ part of the topology. Assume such is true for all $n$ sized collections. Consider an $n+1$ sized collection $\{U_1,U_2 \cdots U_{n+1}\}$ and consider $x \in (\cap_{i=1}^{n} U_{i} ) \cap U_{n+1}$. From the $n=2$ case and induction hypothesis, we are done. Hence, closed under finite intersections. $\tau_{\C}$ is indeed a topology.   }
\thmp{Let $\C$ be a basis for $X$. The topology generated by $\C$, that is, $\tau_{\C}$ is precisely the topology obtained by taking all possible, arbitrary union's of $\C$.}{Consider an arbitrary union of elements $C_{\alpha} \in \C$, $\cup_{\alpha} C_{\alpha}$. $x \in \cup C \implies \exists \alpha$ such that $x \in C_{\alpha}$. Therefore, $\cup_{\alpha}C_{\alpha}$ is in $\tau_{\C}$. Now, let $U \in \tau_{\C}$. For every $x \in U$, there exists $C_x$ in $\C$ such that $x \in C_x \subseteq U $ (Definition of the generated topology). That means, $U=\cup_{x: x \in U} C_x$ which makes it the union of sets in the collection $\C$. We are done.}
\begin{example}
    If $X$ is a metric space, the collection $\{B_{\varepsilon}(x): x \in X, \varepsilon \in \RR^+\}$ is a basis for $X$. 
\end{example}

\begin{remark}
Note that, every topology $\tau$ is its own basis. It must also be noted that the same topology can be generated by two different bases. Consider for example the set of all open subsets of $\RR$. Also consider the set of all open balls in $\RR$. Or, consider the set of all $\frac{1}{n}$ balls around all rational points of $\RR$
\end{remark}
\thmp{(Going from a topology to a basis) Suppose $(X, \tau)$ is a topological space, and $\C$ is a collection of sets in $X$ (which is a subset of $\tau$) such that for every $x \in X$ and for every $U \in \tau$ such that $x \in U$, $\exists C_x \in \C$ such that $x \in C_x \subseteq U$. Then $\tau$ is actually generated by $\C$}{Let $x \in X$. That means, $\exists C_x \in \C$ (from hypothesis) such that $x \in C_x \subseteq X$. Moreover, let $x \in C_1$ and $C_2$. By hypothesis, all $C \in \C$ are conidered open sets, therefore, $C_1 \cap C_2$ is an open set, and $x \in C_1 \cap C_2$. That means $\exists C_3 \in \C$ such that $x \in C_3 \subseteq C_1 \cap C_2$. That makes $\C$ a basis. Its now obvious that $\tau=\tau_{\C}$ }
\thmp{(Criteria for derermining coarseness from basis) Let $\tau$ and $\tau'$ be two topologies of $X$. Also suppose $\tau$ is generated by $\C$ and $\tau'$, by $\C'$. Then the following are equivalent: 
\begin{enumerate}
    \item $\tau'$ is finer than $\tau$
    \item for each $x \in X$ and each basis element $C \in \C$ containing $x$, there exists $C' \in \C'$ such that $x \in C' \subseteq C$

\end{enumerate} }{$\implies)$ Suppose $\tau'$ is finer than $\tau$, that means every open set open in $\tau$ is also open in $\tau'$. Let $C \in \C$, which is open in $\tau$ (and hence, in $\tau'$ too) and let $x \in C$. Since $C$ is open in $\tau'$, and $x \in C$, there exists a basis element of $\tau'$, called $C'$ such that $x \in C' \subset C$. 
\\\\ $\impliedby)$ Let it be that for every $x \in X$ and for every $C \in \C$ such that $x \in C$, there exists $C' \in \C'$ such that $x \in C' \subseteq C$. By the definition of topology generated by bases, it is clear that $\C$ is part of the topology of $\tau'$. Consider $U \in \tau$. 
\\\\ $U$ is an arbitrary union of elements of $\C$ (by the alternate characterisation of generated topology), making $U$ an element of $\tau'$. We are done.
\\\\ (aliter) Let $U$ be such that for every $x \in U$, $\exists C \in C$ such that $x \in C \subseteq U$. Since we are told that there exists $C' \subseteq C$ so that $x \in C'$, we have that for all $x \in U$, $\exists C'_x \in \C' $ so that $x \in C_x' \subseteq U$, making $U$ part of the topology of $\tau'$. }
\defn{Standard topology of $\RR$}{Let $\B$ be the collection of all open intervals of the kind $(a,b):a,b \in \RR$. That this is a basis is obvious. Define the standard topology as that topology generated by this basis.}
\defn{lower limit topology $\RR_l$}{Conider $\B'$ the collection of all sets of the kind $[a,b): a,b \in \RR$. Again, that this is a basis is obvious. Define the lower limit topology as the topology generated by $B'$, and denote it as $\RR_l$}
\defn{K topology $\RR_K$}{Let $K$ be the set of all numbers of the kind $1/n$ for $n \in \ZZ_+$. Denote $\B''$ as the collection of all open intervals of the kind $(a,b)$ along with the sets of the form $(a,b)-K$. Again, that this is a basis is obvious. The topology generated by $\BB''$ is called the $K$ topology, denoted by $\RR_K$}
\thmp{$\RR_K$ and $\RR_l$ are strictly finer than $\RR$, but $\RR_K$ and $\RR_l$ are not comparable.}{Consider $B \in \B$, of the form $(a,b)$. This is also in $\B'$, and a smaller set of the kind $[c,d)$ lies inside $(a,b)$, which means that, if $x \in (a,b)$, there exists $B' \in \B'$ and $B'' \in \B''$ so that $x \in B' \subset (a,b)$ and $x \in \B'' \subset (a,b)$. Moreover, since both $\RR_l$ and $\RR_K$ have elements that are not in $\RR$, it is obviously strictly finer. (Consider $a \in [a,b)$, there is no element of the kind $(x,y)$ that is inside this, that contains $a$. )
\\\\ Consider an element $B=[0,1)$ where we choose $x=0$. Clear that no element from $\RR_K$ can come inside $B$ whilst containing $0$. On the other hand, consider $0$ inside $(-1,1)-K$ in $\RR_K$. No interval of the kind $[a,b)$ can contain $0$ and itself be contained inside $(-1,1)-K$. Hence, it is clear that $\RR_K$ and $\RR_l$ are incomparable.  }

\noindent\fbox{%
    \parbox{\textwidth}{%
        \textbf{(How Dugundji goes about developing an idea of basis)}
        \\\\ Let $\TTT_X$ be the topology of a space $X$. We call $\B \subset \TTT_X$ a \emph{basis} if every open set in $\TTT_X$ is a union of elements of $\B$. (and by definition, since $\B$ is asked to be a subset of $\TTT_X$, we see that every $B \in \B$ is also open. So we can safely say every open set in $\TTT_X$ is an arbitrary union of $\B$, and conversely every arbitrary union of elements of $\B$ belongs in $\TTT_X$)
        \\\\ \textbf{Theorem: } Let $\B \subseteq \TTT_X$. Then, the following are equivalent:\begin{enumerate}
        \item $\B$ is a basis for $\TTT_X$
        \item For every $G \in \TTT_X$, and for every $(x \in X$, $x \in G) \implies x \in B_x \subset G$ for some $B_x \in \B$
        \end{enumerate}
         \begin{proof}$\implies$) Let $\B$ be a basis for $\TTT_X$. Let $x \in G$ for some $G \in \TTT_X$. Since $G=\cup_q B_q$ for $B_q \in \B$, we have that there exists atleast one $B_q$ so that $x \in B_q \subseteq G$.
        \\\\ $ \impliedby)$ Let $U$ be an open set in $\TTT_X$. We are told that for every $x \in U$, there is an element of $\B$ so that $x \in B_x \subseteq U$. This implies $U$ is an arbitrary union of elements of $\B$. This means that $\B$ forms a basis for $\TTT_X$.
        
        \end{proof}
    }%
}

\defn{Subbasis}{A subbasis $S$ is a collection of subsets of $X$ whose union amounts to $X$}
\defn{Topology generated by a subbasis}{$\tau_S$, the topology generated by the subbasis $S$ is defined as the set of all arbitrary unions of the set of all finite intersection of elements of $S$. }
\thmp{$\tau_S$, the topology generated by a subbasis $S$, is indeed a topology.}{Let $S$ be the set whose union forms $X$ (Subbasis). 

Let $Y=\{\gamma: \gamma=\cap_{i=1}^n S_i \text{ for some }n \text{ and }S_i \in S\}$, i.e, $Y$ is the set of all finite intersections of elements of $S$. Consider the set of all arbitrary union of elements of Y, and call it $\tau_S$. Obviously $\tau_S$ cotains $\varnothing$ and $X$. Let $\{U_{\alpha}\}$ be a collection of open sets in $\tau_S$, i.e, each $U_{\alpha}$ is an arbitrary union of elements that are finite intersection of elements of $S$. The union of all $U_{\alpha}$, therefore, would still be an arbitrary union of finite intersections of elements of $S$, hence, is back in $\tau_S$. Now consider $\{U_1,U_2 \cdots U_k\}$, elements of $\tau_S$ such that each $U_i$ is an arbitrary union of a collection of sets that are each a finite intersection of elements of $S$. Consider the finite intersection $\cap_{i=1}^{n} U_i=\cap_{i=1}^n (\cup_{\alpha} (\cap_{j=1}^{n_{\alpha}}S_{j,\alpha}) )$ which is ultimately an arbitrary union of a finite intersection of sets in $S$, making this part of $\tau_S$. Hence, $\tau_S$ is indeed a topology.} 
\thmp{Let $\{\tau_{\alpha}: \alpha \in A\}$ be a collection of topologies of $X$, then there is a unique "smallest" topology $\tau_S$ such that $\tau_S$ contains all the topologies $\{\tau_{\alpha}\}$ (kinda like a LUB), and there is a unique "largest: topology $\tau_L$ that is contained in all these topologies $\{\tau_{\alpha}\}$.}{\underline{\textbf{$\tau_S$ that contains all topologies $\{\tau_{\alpha}\}$}}:
\\\\ \emph{Uniqueness:} If two such topologies exists, we can take the intersection of them to result in a smaller topology (intersection is still a topology) that contains $\{\tau_{\alpha}\}$.
\\\\ \emph{Existence:} Consider the topology generated by arbitrarily unioning the sets generated by finitely intersecting all elements of $\{\tau_{\alpha}: \alpha \in A\}$. Explicitly, we collect $\R=\{U: \exists x \text{ such that }U \in \tau_{x}\}$ and we take all possible finite intersections of elements in $\R$ and collect that in a set $\S$. Then, we take all possible arbitrary unions of elements of $\S$ and call it $\tau_S$. 

For later use: 

$\bullet$ $\R=$ all the open sets in all the topologies. 

$\bullet$ $\S=$ finite intersection of all elements of $\R$

$\bullet$ $\tau_S=$ arbitrary union of all elements of $\S$

Is $\tau_S$ a topology? $X$ and $\varnothing$ are obviously there, since $X$ and $\varnothing$ are there in $\R$, and we take the singleton intersection of these in $\S$, and finally, we take the singleton union of these to arrive at $X, \varnothing \in \tau_S$. Let $\{G_{\delta}:\delta \in \Delta\}$ be a collection of sets in $\tau_S$ (i.e, a collection of sets that are an arbitrary union of a collection of finite intersection of elements of $\R$). $\cup_{\delta}G_{\delta}$ would also be an arbitrary union of elements of $\S$ making it part of $\tau_S$. Let $\{G_1,G_2,\cdots G_k\}$ be in $\tau_S$. $\cap_{i=1}^k G_i=\cap_{i=1}^k (\cup_{\delta_i} (\cap_{j=1}^{n_{\delta_i}} R_{\delta_i,j} ) )$ which, from De Morgan's Laws, can easily be seen to be an arbitrary union of a finite intersection of elements from $\R$. Hence, the intersection is also in $\tau_S$ making it a topology that contains all of $\{\tau_{\alpha}\}$.
\\\\ \underline{\textbf{$\tau_L$ that is contained in all topologies $\{\tau_{\alpha}\}$}}
\\\\ \emph{Existence:} Consider $\cap_{\alpha} \tau_{\alpha}$, which is a topology that is contained in all of $\tau_{\alpha}$. 
\\\\ \emph{Uniqueness:} If two topologies are such that they are contained in every $\tau_{\alpha}$, then they are indeed contained in $\cap_{\alpha} \tau_{\alpha}$ }
\thmp{Let $A$ be a basis for $\tau$, then $\tau_A$, the topology generated by $A$ (The one generated by taking arbitrary union of elements of $A$) is precisely the intersection of all topologies $\{\tau_{\alpha}:\alpha \in G\}$ that contain $A$}{$\tau_A$ is the topology generated by arbitrary union of elements of $A$. We are told $A \subseteq \tau_{\alpha}$ for all $\alpha \in G$. Let $U \in \tau_A$. $U$ is an arbitrary union of elements of $A$, which, since $\tau_{\alpha}$ is a topology, is also contained in $\tau_{\alpha}$. Every open set in $\tau_A$ is also in $\tau_{\alpha}$. Hence $\tau_A \subseteq \cap_{\alpha} \tau_{\alpha}$. 
\\\\ Note that, $\tau_A$ is iself a topology containing $A$, which means $\tau_A=\cap_{\alpha} \tau_{\alpha}$ }
\thmp{Let $A$ be a subbasis for $X$. The topology $\tau_A$ generated by $A$, i,e, by taking arbitrary union of elements that are finite intersection of $A$, is precisely the intersection of all topologies $\{\tau_{\alpha}\}$ that contain $A$.}{Obviousy $\tau_{A}$ contains $A$, since each element of $A$ is a singleton union of the signleton intersection of itself. Moreover, let $U$ be any open set in $\tau_A$, i.e, $U$ is an arbitrary union of a collection of finite intersections of elements of $A$. Let $\tau_{\alpha}$ be a topology that contains $A$. That means a finite intersection of elements of $A$ is still in $\tau_{\alpha}$, and an arbitrary union of these are still in $\tau_{\alpha}$. Hence, $U$ is in $\tau_{\alpha}$, for any $\alpha$. Hence, $\tau_A \subseteq \tau_{\alpha}$ for every $\alpha$, implying $\tau_A \subseteq \cap_{\alpha} \tau_{\alpha}$. Since $\tau_A$ is part of $\{\tau_{\alpha}\}$, we see $$\tau_A =\cap_{\alpha}\tau_{\alpha} $$ where $\{\tau_{\alpha}\}$ is the collection of all topologies that contain $A$.}
\noindent\fbox{%
    \parbox{\textwidth}{%
        \textbf{(How Dugundji goes about developing subbasis)}
        \\\\ \textbf{Theorem:} Given any collection of subsets of $X$, $\Sigma=\{A_{\alpha}:\alpha \in \AA\}$, there exists a unique, smallest topology $\TTT(\Sigma)$ that contains $\Sigma$ as a subset. Moreover, this has a characterisation: It is the collection of all the arbitrary unions of finite intersections of elements of $\Sigma$.
        \begin{proof}
            \textbf{Uniqueness:} If two such topologies $\TTT_1$ and $\TTT_2$ exists, then $\TTT_1 \cap \TTT_2$ becomes a smaller topology that contains $\Sigma$. 
            \\\\ \textbf{Existence:} Take all elements of $\Sigma$ and take all possible finite intersections, and collect them in the set $\R$. Then take all possible arbitrary union of elements of $\R$ and collect them in a set $\S$. This would be a topology, since arbitrary union of an arbitrary union of finite intersections is an arbitrary union of finite intersections. Moreover, finite intersections of arbitrary unions of finite intersections are ultimately arbitrary union of finite intersections. We denote this topology by $\TTT(\Sigma)$.
            \\\\ Consider for a moment the set of all topologies $\TTT_{\gamma}$ that contains $\Sigma$. $\cap_{\gamma}\TTT_{\gamma}$, intersection of all topologies that contain $\Sigma$, is indeed a topology, as is easy to verify. Moreover, it is the smallest topology that contains $\Sigma$; This much is evident.
            \\\\ Let $U$ be an arbitrary union of a finite intersection of elements of $\Sigma$. Since $\cap_{\gamma} \TTT_{\gamma}$ contains $\Sigma$, it contains $U$ as well, implying that $\TTT(\Sigma) \subseteq \cap_{\gamma} \TTT_{\gamma}$. Since $\TTT(\Sigma)$ is itself a topology that contains $\Sigma$, we see that $\TTT(\Sigma)=\cap_{\gamma} \TTT_{\gamma}$, i.e
            $\TTT(\Sigma)$ is the smallest topology that contains $\Sigma$.


        
        \end{proof}
        \textbf{Definition:}(Topology generated by a family $\Sigma$, $\TTT(\Sigma)$) The topology generated by any family of subsets of $X$ is defined as $\TTT(\Sigma)$, the collection of all arbitrary union of finite intersections of elements of $\Sigma$.
    }%
}

\subsection{Order Topology}
\defn{Simple Order}{Let $X$ be a set. Define $C:X \times X \to \{y,n\}$ such that: \begin{enumerate}
    \item If $x \neq y$, either $xCy$ or $yCx$
    \item For no $x$ is it true that $xCx$
    \item if $xCy$ and $yCz$, then $xCz$
\end{enumerate} This type of relation is called \emph{simple order} relation.
\\\\ \textbf{Pro Tip:} For intuition purposes, replace $C$ with $<$. }

\defn{Order Topology on a space $X$.}{Let $X$ be a space with more than two points, equipped with a simple order relation $<$. Consider the collection $\B$ of sets of the following kind:
\begin{enumerate}
    \item $(a,b)$ where $a<b$. i.e, all open intervals in $X$
    \item $[a_0,b)$ for all $b \in X$, where $a_0$ is the smallest element of $X$
    \item $(a,b_0]$ for all $a \in X$ where $b_0$ is the largest element in $X$
\end{enumerate} Then the order topology $\tau_O$ is defined as that topology generated by $\B$.
\begin{remark}
    Note that, if $X$ has no smallest element, type (2) sets won't exist in $\B$. Likewise, if $X$ has no largest element, type (3) sets won't exist in $\B$.

\end{remark}}
\thmp{$\B$ is actually a basis (hence, $\tau_O$ is actually a well defined topology on $X$)}{Let $x \in X$. If $x$ is not the largest element, nor the smallest, then there exists $(a_0,b_0)$ that contains $x$
\\\\ If $x$ is the largest element, then $(a_0,x]$ contains $x$. If $x$ is the smallest element, then $[x,b_0)$ contains $x$. Hence, for every $x \in X$, there exists an element $B$ from $\B$ such that $x \in B$. 
\\\\ Note the following, if we intersect two type 1 sets, we get back a smaller type 1 set. i.e, $(a,b) \cap (a',b'):= \{x\in X: a<x<b \text{ and } a'<x<b' \}$.
\\ If we intersect two type 2 sets, we get back a type 2 set (obviously)
\\ If we intersect two type 3 sets, we get back a type 3 set (obviously)
\\ If we intersect a type 1 and type 2 set, i.e $(a,b)$ and $[a_0,x)$, we are basically saying $\{t \in X: a < t < b \text{ and }a_0 \leq t<x \}$ We get back a type 1 set.
\\ If we intersect a type 1 and type 3 set, i.e $(a,b)$ and $(x,b_0]$, we get $\{t \in X: a<t<b \text{ and } x<t \leq b_0 \}$ which would yield a set of type 1.
\\ If we intersect a type 2 and type 3 set, i.e, $[a_0,x)$ and $(y,b_0]$, we want basically $\{t \in X: a_0 \leq t <x \text{ and } y<t \leq b_0\}$. It is easy to see that this intersection falls into one of the 3 categories. 
\\\\ Hence, $\B$ forms a basis, and hence, $\tau_B$ is a topology.
}
\begin{example}
    The standard topology $\RR$ is basically the order topology on $\RR$. All one needs to see is that there is no least element, and largest element of $\RR$ (Of course, $\RR \cup \{-\infty,+\infty\}$, the extended real line, \emph{does} have these). Hence, $\B$ is just all elements of the kind $(a,b)$, which is precisely the basis for the standard topology.  
\end{example}
\defn{Rays}{Given $a \in X$ (a simply ordered set), there are 4 sets associated with $a$ called \emph{rays}:
\begin{enumerate}
    \item $(a,+\infty):=\{x: a<x\}$
    \item $(-\infty,a):=\{x: x<a\}$
    \item $[a,+\infty):=\{x:a \geq x\}$
    \item $(-\infty,a]:=\{x: x \leq x\}$

\end{enumerate}}
\thmp{The first two rays are open in the order topology, (and the 2nd two rays are closed in the order topology, i.e, its complement is open)}{Consider $(a,+\infty)$. If there is no maximum element, then this is precisely an arbitrary union of sets of the kind $(a,b_{\alpha})$. If there is a maximal element, then this is $(a,b_0]$. In a similar fashion, the 2nd set is also open.
\\\\ Consider $[a,+\infty)^C=(-\infty,a)$, and consider $(-\infty,a]^C=(a,\infty)$. Obvious from here.}
\thmp{The collection of all open \emph{rays} forms a subbasis for the order topology on $X$}{We have to show that the topology generated by arbitrarily unioning a collection of finite intersections of rays gives rise to the topology generated by the bais of order topology of $X$. Let $\tau_R$ be the topology generated by the arbitrary union of finite intersection of rays. Let $\tau$ be the order topology on $X$, generated by arbitrary union of the basis elements of $X$. We want to show $\tau_R \subseteq \tau$ and $\tau \subseteq \tau_R$.
\\\\ Consider $(a,b)$, a basis element of $\tau$. $(a,+\infty) \cap (-\infty,b)$ is a finite intersection of rays, and yields the basis $(a,b)$. $[a_0,b)$ is, simply, $(-\infty,b)$ and $(a,b_0]$ is simply $(a,+\infty)$. Hence, finite intersection of the rays yield the basis elements. Arbitrary union of finite intersection of rays can yield all the possible open sets of the order topology on $X$. Hence, $\tau \subseteq \tau_R$. Let $S$ be open in $\tau_R$. It is the arbitrary union of a finite intersection of rays. Note that every ray is either an element of the basis for $\tau$ or is an arbitrary union elements of basis of $\tau$. An arbitrary union of a finite intersection of an arbitrary union of elements of the basis yields an arbitrary union of a finite intersection of elements of the basis, giving rise to yet another element of the topology $\tau$. Hence, every open et in $\tau_R$ is open in $\tau$, hence $\tau_R \subseteq \tau$ }
\subsection{The product topology}
\defn{Product topology on $X \times Y$}{Let $\tau_X$ and $\tau_Y$ be topologies of $X$ and $Y$ respectively. Let $\B$ be the collection of subsets of $X \times Y $ of the kind $U \times V$ for $U \in \tau_X$ and $V \in \tau_Y$. We define the product topology as that which is generated by the basis $\B$. }
\thmp{$\B$ of the previous definition is indeed a basis.}{Let $(x,y) \in X \times Y $. Since $X \in \tau_X$ and $Y \in \tau_Y$, we have that $(x,y)\in X \times Y$ so the 1st condition for basis is satisfied. Let $(x,y) \in U_1 \times V_1$ and $U_2 \times V_2$. Consider $(U_1 \times V_1)\cap (U_2 \times V_2)$. This is same as $U_1 \cap U_2 \times V_1 \cap V_2$ which is again an element of $\B$. Hence, it is a basis and hence $\tau_{\B}$ is well defined.}
\thmp{If $X'$s topology $\TTT_X$ is generated by a basis $\B_X$ and $Y'$s topology $\TTT_Y$ is generated by a basis $\B_Y$, then the product topology of $X \times Y$ 

(given by$\langle \{U \times V: U \text{ open in} \TTT_X \text{ and } V \text{ open in } \TTT_Y \} \rangle$) has a basis $\B:=\{C \times D: C \in \B_X, D \in \B_Y\}$, i.e, $$\TTT_{X \times Y}=\langle \{U \times V: U \in \TTT_X, V \in \TTT_Y\} \rangle=\langle_{\text{basis}} \B:=\{C \times D: C \in \B_X, D \in \B_Y\} \rangle$$}{\emph{Checking that $\B$ is indeed a basis:}
\\\\ Let $(x,y) \in X \times Y$. $x \in X$ means that there exists $B \in \B_X$ so that $x \in B \subseteq X$. Likewise, for $y$, there is a $C \in \B_Y$ so that $y \in C \subseteq Y$, which means $(x,y) \in B \times C \subseteq X \times Y$. This means that $\B$ obeys the 1st condition for a basis.
\\\\ Let $(x,y) \in B,B' \in \B$. i.e, $(x,y) \in C_1 \times D_1, \text{ and }C_2 \times D_2 $ for $C_1,C_2 \in \B_X$ and $D_1,D_2 \in \B_Y$. That means that $(x,y) \in (C_1 \times D_1) \cap (C_2 \times D_2)=(C_1 \times C_2) \times (D_1 \times D_2)$. $x \in C_1 \cap C_2$ means there is a $C_3$ so that $x \in C_3 \subseteq C_1 \cap C_2$. Likewise, $y \in D_3 \subseteq D_1 \cap D_2 $. Hence, $(x,y) \in (C_3 \times D_3) \subseteq (C_1 \cap C_2) \times (D_1 \times D_2)=(C_1 \times D_1)\cap (C_2 \times D_2)$. Hence, $\B$ is a basis.
\\\\ Now consider any open set in $\TTT_{X \times Y}.$ Since $\TTT_{X\times Y}$ is generated by $U \times V: U \in \TTT_X, V \in \TTT_Y$, Every open set in $\TTT_{X \times Y}$ is an arbitrary union of sets of the kind $U \times V$, i.e, $\cup_{\alpha}(U_{\alpha}\times V_{\alpha})$. But each $U_{\alpha}= \cup_{\delta(\alpha)} B_{\delta}^{\alpha}$ for $B_{\delta}^{\alpha} \in \B_X$. Likewise, each $V_{\alpha}= \cup_{\delta(\alpha)} C_{\delta}^{\alpha} $ for  $C_{\delta}^{\alpha} \in \B_Y$. Let $(x,y)$ be any point in $X$, that is in $\cup_{\alpha}(U_{\alpha} \times V_{\alpha})$ which means that there exists $\alpha_0$ so that $(x,y) \in U_{\alpha_0} \times V_{\alpha_0}= \cup_{\delta(\alpha_0)} B_{\delta} \times \cup_{\delta{(\alpha_0)}} C_{\delta}$. This means $x \in B_{\delta_0}$ and $y \in C_{\delta_1}$ for some indices. This means that $(x,y) \in B_{\delta_0} \times C_{\delta_1}$, which is part of the basis $\B$. Hence, Every open set in $\TTT_{X\times Y}$ is open in $\langle \B \rangle$ (or $\TTT_{X \times Y} \subseteq \langle \B \rangle$). Moreover, it is obvious to see that every element of $\B$ is open in $\TTT_{X \times Y}$. Hence, arbitrary union of elements of $\B$, which are precisely the open sets of $\B$, are also open in $\TTT_{X \times Y}$. Hence, $\TTT_{X \times Y}=\langle \B \rangle$. 
}
\begin{example}
    Consider $X=\RR, Y=\RR$, with the standard topologies (i.e, that which is generated by the basis elements of the kind $\{(a,b); a<b\}$). The \textbf{\emph{standard topology}} on $\RR \times \RR=\RR^2$ is defined as the product topology. A basis for this topology is the set of all $U \times V$ for $U$ and $V$ open in $\RR$, but from the preceeding result we can find an even smaller subset that makes a basis: $\{B \times C: B \in \B_X, C \in \B_Y\} $ which is precisely $\{(a,b) \times (c,d): a,b,c,d \in \RR; a<b, c<d\}$
\end{example}
\defn{Projections}{Define $\pi_1:X \times Y \to X$ by $\pi_1(x,y)=x$ and $\pi_2:X\times Y \to Y$ given by $\pi_2(x,y)=y$. These maps are called projection maps.}
\thmp{The projection maps are onto, and have the property that, if $U$ is an open subset of $X$, and $V$ is an open subset of $Y$, then $\pi_1^{-1}(U)=U \times Y $ and $\pi_2^{-1}(V)=X \times V$ (well, open hypothesis is not needed really)}{That they are onto functions is obvious. Let $U$ be an open subset of $X$. Consider $\pi_1^{-1}(U):=\{(x,y) \in X \times Y: \pi_1(x,y)=x \in U\}$. This is precisely $U \times Y$. Similarly, $\pi_2^{-1}(V)=X \times V$. Note that, in the product topology, $U \times Y$ and $X \times V$ are both open sets. Moreover, the intersection of $\pi_1^{-1}(U)$ and $\pi_2^{-1}(V)$ is $(U \times Y) \cap (X \times V)= U \times V$ }
\thmp{Let $\TTT_X$ be a topology for $X$ and $\TTT_Y$ for $Y$. Let $\TTT_{X \times Y}:=\langle \{U \times V: U \in \TTT_X, V \in \TTT_Y\} \rangle $ be the product topology on $X \times Y$. Then the set $\Gamma:=\{\pi_1^{-1}(U):U \in \TTT_X\} \cup \{\pi_2^{-1}(V): V \in \TTT_Y\}$ forms a subbasis for $\TTT_{X\times Y}$}{$\Gamma$ is full of elements of the kind $U \times Y$ and $X\times V$ (every element is of this kind, and every element of this kind is in $\Gamma$). Consider any open set in $\TTT_{X \times Y}$, which is an arbitrary union of elements of the type $U \times V$, or explicitly, an open set looks like $\cup_{\alpha}(U_{\alpha} \times V_{\alpha})$ which is the same as $\cup_{\alpha}((U_{\alpha} \times Y)\cap (X \times V_{\alpha}))=\cup_{\alpha} \pi_{1}^{-1}(U_{\alpha}) \cap \pi_2^{-1}(V_{\alpha})$ which is an arbitrary union of an intersection of elements of $\Gamma$. Hence, we see that every set open in $\TTT_{X \times Y}$ is open in $\langle_{subbasis} \Gamma\rangle$, which means $\TTT_{X \times Y} \subseteq \langle_{subbasis }\Gamma \rangle$. Let $C$ be a set in $\langle_{subbasis} \Gamma \rangle$, i.e, $C$ is an arbitrary union of a finite intersection of sets in $\Gamma$. If the finite intersection only contains elements of the kind $U \times Y$ (or $X \times V$), then the intersection would ultimately be of the kind $U_0 \times Y$ for some $U_0$ open in $\TTT_X$ (or $X \times V_0$ for some $V_0$ open in $\TTT_Y$), which would then mean $\cup_{0} U_0 \times V$ would finally still yield a set of the form $U \times V$, falling again into $\TTT_{X \times Y}$ (or the other thing). 
\\\\ If suppose the finite intersection contains both elements of the type $U \times Y$ and $X \times V$, then intersecting them yields $U \times V$, and suppose some other set is of the kind $U' \times Y$ (or $X \times V'$), an intersection with them still yields a set of the type $U'' \times V''$ for some open $U'' \in \TTT_X$ and $V'' \in \TTT_Y$. This again shows that, arbitrarily unioning them will still make it fall into $\TTT_{X \times Y}$, which finally gives us $\langle_{subbasis} \Gamma \rangle=\TTT_{X \times Y}$ }
\subsection{Subspace Topology}
\defn{Subspace Topology}{Let $X$ be a topological space with a topology $\TTT_X$ and let $Y \subset X$. We define the \emph{\textbf{subspace topology}} on $Y$ as $\TTT_Y:=\{U \cap Y: U \in \TTT_X\}$}
\thmp{The Subspace topology is indeed a topology on $Y$}{$\varnothing=Y \cap \varnothing$ and $Y=Y \cap X$. Hence $\varnothing$ and $X$ are both in $\TTT_Y$. Let $U_{\alpha}$ be a collection of sets in $\TTT_Y$, i.e $U_{\alpha}=G_{\alpha} \cap Y$ for $G_{\alpha} \in \TTT_X$. $\cup_{\alpha} U_{\alpha}=\cup_{\alpha} (G_{\alpha} \cap Y)=(\cup_{\alpha} G_{\alpha}) \cap Y \in \TTT_Y$. Closed under arbitrary union.
\\\\ Let $U_1,U_2 \cdots U_k$ be a bunch of sets of the kind $G_i \cap Y$. Arbitrary intersection of these yields $\cap_{i=1}^k U_i= \cap_{i=1}^{k} (G_i \cap Y)= (\cap_{i=1}^{k} G_i) \cap Y$ which is again of the form $K \cap Y$ for some $K \in \TTT_X$, which makes this finite intersection a part of $\TTT_{Y}$. Hence, $\TTT_Y$ is a topology.   }
\lemp{Let $\B_X$ be a basis for $\TTT_X$ on $X$. Then the basis of $\TTT_Y$, the subspace topology of $Y \subseteq X$ is $\C:=\{B \cap Y: B \in \B_X\}$}{Let $y \in Y$, this means $y \in X$, which means $y \in B \in \B_X$. That means $y\in B \cap Y \in \C$. Let $y \in C_1$ and $C_2$, which are in $\C$. That means $y \in (B_1 \cap Y) \cap (B_2 \cap Y)=(B_1 \cap B_2) \cap Y$ which means $y \in (B_3) \cap Y$ so that $B_3 \subseteq B_1 \cap B_2$, implying that $B_3 \cap Y \subseteq B_1 \cap Y \cap B_2 \cap Y$, giving us that if $y \in C_1$ and $C_2$ in $\C$, there exists $C_3 (=B_3 \cap Y)$ so that $y \in C_3 \subseteq C_1 \cap C_2$, hence $\C$ forms a basis for $\TTT_Y$. Now we need to show that $\TTT_{Y}:=\{U \cap Y: U \in \TTT_X\}$ is the same as $\TTT_{\C}:=\langle \C \rangle$. Let $R$ be open in $\TTT_{Y}$ implying it is of the form $U \cap Y$ for some $U \in \TTT_X$. But $U= \cup_{\alpha} B_{\alpha}$ for basis elements $B_{\alpha} \in \B$. That gives $R=U \cap Y=(\cup_{\alpha} B_{\alpha} )\cap Y=\cup_{\alpha} (B_{\alpha}\cap Y) $ which is an arbitrary union of elements of $\C$. Hence, $\langle \C\rangle \subseteq \TTT_Y$.
\\\\ Let $G$ be in $\langle \C \rangle$, i.e, is an arbitrary union of $C_{\alpha}=B_{\alpha} \cap Y \in \C$. Hence, $G=\cup_{\alpha}(B_{\alpha} \cap Y)= (\cup_{\alpha} B_{\alpha})\cap Y $ making it part of $\TTT_Y$ (its of the form "some open set" intersected with $Y$). This means $\TTT_Y \subseteq \langle \C \rangle$ which ultimately gives $\langle \C \rangle = \TTT_Y$ }
\lemp{Let $Y \subseteq X$ and let $\TTT_Y$ be the subspace topology on $Y$ inherited from the topology $\TTT_X$ on $X$. Then, $Y$ is open in $X$ (i.e, is in $\TTT_X)$ if and only every set open in $Y$, i,e, is in $\TTT_Y$ is open in $X$.}{$\implies$) Let $Y$ be open in $X$, that means $Y \in \TTT_X$. $\TTT_Y:=\{U \cap Y: U \in \TTT_X \}$. Let $G=U \cap Y$ be an open set in $\TTT_Y$, for some open set $U$ open in $X$. Since $\TTT_X$ is closed under finite intersections, $G=U \cap Y$ is open in $\TTT_X$. Hence every set in $\TTT_Y$ is in $\TTT_X$.
\\\\ $\impliedby$) If every set in $\TTT_Y$ is in $\TTT_X$, that means that $Y$ is also in $\TTT_X$, making $Y$ open in $X$.}
\thmp{Let $X$ be a space with a topology $\TTT_X$ and $Y$ with $\TTT_Y$. Consider $A \subseteq X$ and $B \subseteq Y$. $A \times B \subseteq X \times Y$. We have that $$\TTT^{S}_{A\times B}=\TTT^{P}_{A \times B}$$ where $\TTT^{S}_{A \times B}$ represents the subspace topology of $A \times B$ viewed as a subspace of the product topology $X \times Y$, and $\TTT^{P}_{A \times B}$ represents the product topology of $A \times B$, whose individual topologies come from the corresponding subspace topologies.}{Let $F$ be an open set in $\TTT^{S}_{A \times B}$, i.e, it is of the form $G \cap (A \times B)$ for some $G \in \TTT_{X \times Y}$, i.e, $G$ is an arbitrary union $\cup_{\alpha} (U_{\alpha} \times V_{\alpha})$ ($U_{\alpha} \in \TTT_X$ and $V_{\alpha} \in \TTT_Y$). Therefore, $F=[\cup_{\alpha}(U_{\alpha}\times V_{\alpha})]\cap(A \times B)= \cup_{\alpha} ((U_{\alpha} \times V_{\alpha}) \cap (A \times B))=\cup_{\alpha} ((U_{\alpha} \cap A) \times (V_{\alpha} \cap B))$. Note that $U_{\alpha} \cap A$ and $V_{\alpha} \cap B$ are each open in $A$ and $B$ respectively (from the subspace topology inherited from $\TTT_X$ and $\TTT_Y$ respectively). Therefore, $U_{\alpha} \cap A \times V_{\alpha} \cap B$ is part of the basis for the product topology $A \times B$ (since its a product of open sets in $A$ and $B$). Hence, $F$ is an arbitrary union of elements of $\{U \times V: U \in \TTT_A, V \in \TTT_B\}$ which means that it is in $\TTT^{P}_{A \times B}$. This implies $\TTT^{S}_{A \times B} \subseteq \TTT^{P}_{A \times B}$.
\\\\ Let $F'$ be an open set in $\TTT^{P}_{A \times B}$, i.e, is an arbitrary union of elements of the kind $U_r \times V_r$ where $U_r \in \TTT_A$ and $V_r \in \TTT_B$. $F'= \cup_{r} (U_r \times V_r)=\cup_{r}(G_r \cup A) \times (E_r \cup B)$ for $G_r \in \TTT_X$ and $E_r \in \TTT_Y$. This is $F'=\cup_{r} (G_r \times E_r) \cup (A \times B)= \cup_{r} (\text{basis element of }\TTT_{X \times Y}) \cap (A \times B)= \cup_{r} (\text{some basis element of }A\times B)_{r} $ which is obviously open in $\TTT^{S}_{A \times B}$. Hence, $\TTT^P_{A \times B} \subseteq \TTT^S_{A \times B }$}
Though the subspace topology inherited by $A \times B \subseteq X \times Y$ is precisely the product topology of the corresponding subspace topologies, a similar result, though reasonable to expect for subspaces of order topologies, does not, unfortunately, hold; as the following example reveals:
\begin{example} (\textbf{Order topology inherited from superspace is \emph{not} equal to the subspace topology inherited from ordered superspace})
    Consider $Y=[0,1] \cup \{2\} \subseteq \RR$. The topology on $\RR$ is generated by the basis $(a,b):a,b \in \RR$. The subspace topology on $Y$ is generated by the sets $(a,b) \cap Y: a,b \in \RR$. Call this $\TTT^S_Y=\langle \{(a,b)\cap Y: a<b\}\rangle$.  The order topology on $Y$ is generated by sets of the kind $(x,y): x,y \in Y$, $[0,y): y \in Y$, along with $(x,2: x \in Y]$. Note that $\{2\}$ is open in the subspace topology since $\{2\}=(1.5,2.5) \cap Y$. But $\{2\}$ cannot be open in the order topology inherited by $Y$ since any basis element containing $2$ contains some other element of $Y$ as well. \end{example}

    \begin{example}
        Let $I=[0,1]$. Consider $I \times I$ as a subset of $\RR\times \RR$, which has topology $\TTT^O_{\RR \times \RR}$ obtained by the simple lexicographic order. The basis elements of $\TTT^0_{\RR \times \RR}$ is given by $\{(a\cdot b, c \cdot d): a<c \text{ or } (a=c \text{ and }b<d )\}$ (Since there is no smallest or largest element of $\RR \times \RR$ in the lexicographic order). The basis for $\TTT^S_{I \times I}$ is $(a\cdot b, c\cdot d: a,b,c,d \in \RR ) \cap (I \times I)$. Meanwhile, the basis for $\TTT^0_{I \times I}$ are sets of the form $\{(p\cdot q, r\cdot s):p,q,r,s \in I\}$, $[0\cdot 0, x \cdot y)$, $(x\cdot y, 1\cdot 1]$. Note that $\{0 \cdot 0\}$ is an open set in $\TTT^0_{I \times I}$ since it is $(0\cdot0, x \cdot y) \cap [0\cdot 0, x \cdot y)$, but no bais element of $\TTT^S_{I \times I}$ can be contained in $\{0 \cdot 0 \}$. Hence, $\TTT^S_{I \times I}$, the topology on $I \times I$ as a subspace of $\RR\times\RR, \TTT^0_{\RR \times \RR}$ is not the same as $\TTT^0_{I \times I}$, the topology on $I \times I$ obtained from the inherited ordering on $I \times I$ from $\RR\times\RR$.
\end{example}
Taking motivation from the previous example, we define the following:
\defn{Ordered Square}{The set $I \times I \subseteq \RR \times \RR$, where $I=[0,1]$ with the lexicographic order topology is called the \emph{ordered square} and is denoted by $I^2_0$ }

\lemp{Let $\B$ be a subbasis for the topology $\TTT$ on $X$. Then the subspace topology $\TTT_S$ of $S \subseteq X$ is generated by the subbasis $\{G \cap S: G \in \B\}$}{Let $U$ be an open set in $\TTT_S$. Then it is of the form $K \cap S$ for some open set $K$ in $\TTT$, making it an arbitrary union $\cup_{\alpha} (\cap_{i=1}^{k_{\alpha}}(E_{\alpha,i}))$ of elements $E_{\alpha,i}$ of $\B$. This gives $U=( \cup_{\alpha} (\cap_{i=1}^{k_{\alpha}}(E_{\alpha,i})) )\cap S= \cup_{\alpha} ((\cap_{i=1}^{k_{\alpha}}(E_{\alpha,i}))\cap S)=\cup_{\alpha} (\cap_{i=1}^{k_{\alpha}}(E_{\alpha,i}\cap S))$ which is an arbitrary union of a finite intersection of elements from $\{G \cap S: G \in \B\}$. Hence, $\langle_{subbasis}\{G \cap S: G \in \B\} \rangle \subseteq \TTT_S$. Any element of $\langle_{subbasis}\{G \cap S: G \in \B\} \rangle$ is of the form $K=\cup_{\alpha} (\cap_{i=1}^{k_{\alpha}}(F_{\alpha,i}))$ for $F_{\alpha,i}=G_{\alpha,i} \cap S$ for $G_{\alpha,i}$ that are in $\B$. Then we can rewrite $K$ as $\cup_{\alpha} (\cap_{i=1}^{k_{\alpha}}(G_{\alpha,i} \cap S))=\cup_{\alpha} ([\cap_{i=1}^{k_{\alpha}}(G_{\alpha,i}))]\cap S$ which is of the form $\gamma \cap S$ where $\gamma$ is an arbitrary union of a finite intersection of elements of $\B$, making $\gamma$ open in $\TTT$ and ultimately $K$ open in $\TTT_S$. Hence, $\langle_{subbasis}\{G \cap S: G \in \B\} \rangle \subseteq \TTT_S$ giving us $\langle_{subbasis}\{G \cap S: G \in \B\} \rangle=\TTT_S$.  }

\defn{Convex Sets}{Given an ordered set $X$, a subset $Y$ is called \emph{\textbf{convex}} if for any two points $a,b$ in $Y$, if $a<b$, then $(a,b)$ is fully contained in $Y$. Intervals and rays are always convex.}
\thmp{Let $X$ be an ordered set in the order topology $\TTT^0_X$. Let $Y$ be a subset of $X$ that is convex, i.e, for any two points $a,b$ in $Y$ so that $a<b$, we have $(a,b) \subseteq Y$. Then the order topology on $Y$, $\TTT^0_Y$ is the same as the subspace topology $\TTT^S_Y$ inherited from the order topology $\TTT^0_X$.}{$$\TTT^0_Y:= \langle \{(x,y)_{Y}: x,y \in Y\}, \{[y_L,q)_Y:q \in Y\}, \{(q,y_U]_Y: q \in Y\} \rangle$$
$$\TTT^S_Y:=\langle \{(a,b)_X \cap Y: a,b \in X\}, \{[a_0,b)_X \cap Y: b \in X\}, \{(a,b_0]\cap Y: a \in X\} \rangle $$ 
First we prove that $\TTT^0_Y \subseteq \TTT^S_Y$. Consider the basis elements of the kind $(x,y)_Y: x,y \in Y$ and let $z \in (x,y)_Y$. $(x,y)_Y:=\{z \in Y: x<z<y\}=\{z \in X: x<z<y \text{ and }z\in Y\}=(x,y)_X \cap Y$ which is a basis element of $\TTT^S_Y$. 
\\\\ Now consider basis elements of the kind $[y_L,y)_Y=\{z \in Y: y_L \leq z <y\}=\{z \in X: y_L\leq z<y \text{ and } z \in Y\}=[y_L,y)_X \cap Y$. if $q \in [y_L, y)_Y$ and $q \neq y_L$, then we can look at $(y_L,y)_X \cap Y$ which contains $q$ and is readily inside $[y_L,y)_Y$. If $q=y_L$, Consider a smaller element that $y_L$, say $L \in X$, and consider $(L,y)_X \cap Y=(L,y)_Y=[y_L,y)_Y$, which implies that $q=y_L \in (L,y)_X \cap Y \subseteq [y_L,y)_Y$ (If $y_L$ is the least in $X$, too, then take the 2nd type basis element of $\TTT^S_X$). 
\\\\ Consider the bases elements of the kind $(y,y_U]_Y=(y,y_U]_X \cap Y$. Let $q$ be in this set, and not equal to $y_U$. Then take $(y,y_U)_X \cap Y$ which is a subset of the above basis element. If $q=y_U$, then in $X$ take a larger element $U$ and consider $(y,U)_X \cap Y=(y,y_U]_Y$ which contains $q$ and is itelf contained in $(y,y_U]_Y$. Moreover, if the largest element of $X$ and $Y$ conincide, then take the 3rd type from $\TTT^S_Y$. 
\\\\ Therefore, $\TTT^0_Y \subseteq \TTT^S_Y$.
\\\\ Now we prove that $\TTT^S_Y \subseteq \TTT^0_Y$. If our basis element is of the kind $(a,b)_X \cap Y$, and $p \in (a,b)_X \cap Y$, we have a few cases to consider. \begin{enumerate}
\item $a< y$ for every $y \in Y$, and $b>y$ for every $y \in Y$. \begin{enumerate}\item Suppose no $z$ is in $Y$ so that $a<z<q$, and likewise, no $z'$ is so that $q<z'<b$. That makes $q$ the least, and the largest element of $Y$. The order topology on $Y$ then would be trivial. \item Suppose such $z$ exists so that $a<z<q$, but no such $z'$ exists so that $q<z'<b$. That means $q$ is the largest element, and we can pick the basis $(z,q]_Y$ that contains $q$ and is inside $(a,b)_X \cap Y$. \item Suppose no such $z$ exists but such $z' $ exists, then we see $q$ is the least element, and thus we can look at $[q,z')_Y$ that contains $q$. \item If suppose both $z$ and $z'$ exist, then we can simply choose $(z,z')_Y$ containing $q$.  \end{enumerate}
\item Suppose $z \in Y$ is so that $z<a$ but $b>y$ for every $y$. \begin{enumerate}
    \item Suppose no $z'$ exists in $Y$ so that $q<z'<b$. Then $q$ would be the largest element of $Y$, and due to convexity of $Y$, $a \in Y$. This would give us the basis element $(a,q]_Y$ that contains $q$ and is fully contained in $(a,b)\cap Y$
    \item If there exists $z'$ in $Y$ so that $q<z'<b$, simply choose the basis element $(a,z')_Y$ that contains $q$ and is contained in $(a,b)_X \cap Y$
\end{enumerate}
\item Suppose $z' \in Y$ is so that $b<z'$ but for every $y \in Y$, $a<y$. \begin{enumerate}
    \item Suppose no $z''\in Y$ exists so that $a<z''<q$, that means that $q$ is the minimal element of $Y$, and that, from convexity, $b$ is in $Y$. This means we can pick the basis element $[q,b)_Y$ that contains $q$ and is contained in $(a,b)_X\cap Y$
    \item Suppose there exists $z'' \in Y$ so that $a<z''<q$. Then we can pick $(z'',b)_Y$ (since $b$ is in $Y$) as our element that contains $q$ and is contained in $(a,b)_X \cap Y$.
\end{enumerate}
\item Suppose $h\in Y$ so that $b<h$ and $k \in Y$ so that $k<a$, we have $(a,b)_X$ completely inside $Y$, which gives us $(a,b)_Y$ as our choice of basis element.
\end{enumerate}
If our basis element is of the kind $[a_0,b)_X \cap Y$ and $q$ is in this set, then we have the following cases:
\begin{enumerate}
    \item If $Y$ has a least element and there exists $z_0\in Y$ so that $q<z_0<b$, then we can choose $[y_0,z_0)_Y$ as that which contains $q$ and is contained in $[a_0,b)_X \cap Y$.
    \item If $Y$ has a least element and there exists no $z$ in $Y$ so that $q<z<b$, then either its possible that some $z'$ exists in $Y$ so that $b<z'$, but then $(q,b)$ would then fully contained in $Y$ so that we can use $[y_0,b)_Y$ as our basis element. If no such $z'$ exists, then $q$ becomes the largest element in $Y$ because $b$ is larger than any element of $Y$. That means we can choose $(y_0,q]_Y$ as our basis element that contains $q$ and is contained in $[a_0,b)_X \cap Y$.
    \item If $Y$ has no least element, then there exists $p \in Y$ so that $a_0<p<q$. \begin{enumerate}
        \item If suppose there exists no $z \in Y$ so that $q<z<b$, then either its possible that some $z'' \in Y$ so that $b<z''$ which, from convexity condition, makes $b$ fall into $Y$, and we can choose our interval to be $(p,b)_Y$ accordingly. So assuming there exists no $z'' \in Y$ so that $b<z''$, we see that $b$ is larger than all elements of $Y$ and no element of $Y$ is larger than $q$. Therefore, $q$ becomes the maximal element of $Y$, and thus, the set $(p,q]_Y$ can be choosen.
        \item If there exists $z \in Y$ so that $q<z<b$, then the interval $(p,z)_Y$ falls into $[a_0,b)_X \cap Y$ and also contains $X$. 
    \end{enumerate}
\end{enumerate}
If we take a basis element of the kind $(a,b_0]_X \cap Y$ and $q$ is in this set, then we have the following cases:
\begin{enumerate}
    \item If $Y$ has a largest element $y_0$ and there exists $z_0$ in $Y$ so that $a<z_0<q$, then we can simply take $(z_0,y_0]_Y$ that contains $q$, and also is contained in $(a,b_0]_X \cap Y$. 
    \item If $Y$ has a largest element $y_0$ but there is no $z_0$ in $Y$ so that $a<z_0<q$, then either there is a $t \in Y$ so that $t<a$ which makes $a$ fall inside $Y$, thus allowing us to make use of $(a,y_0]_Y$. So we suppose $a$ is smaller than all elements in $Y$. This means that $q$ is indeed the smallest element of $Y$, so we can make use of $[q,y_0)_Y$ that contains $q$ and falls inside $(a,b_0]_X \cap Y$.
    \item If $Y$ has no largest element, then there exists $r\in Y$ so that $q<r<b_0$. \begin{enumerate}
        \item If suppose there exists no $z'' \in Y$ so that $a<z''<q$, then either its possible that some $t$ is there in $Y$ that is smaller than $a$, thus making $a$ a point of $Y$ so that we can make use of $(a,r)_Y$ that contains $q$. So we suppose there is no such $t$, i.e, $a$ is smaller than all elements of $Y$. This implies $q$ is the minimal element of $Y$, so that we can make use of $[q,r)_Y$ that contains $q$ and is itself contained in $(a,b_0]_X \cap Y$.
        \item If there exists $z'' \in Y$ so that $a<z''<q$, then we can choose $(z'',r)_Y$ as that interval that contains $q$ such that it is itelf contained in $(a,b_0]_X \cap Y$.
    \end{enumerate} 
In all cases, for every basis element of $\TTT^S_Y$ and for every $x$ in said basis element, there exists a basis element of $\TTT^0_Y$ that contains $x$ and is iteslf contained in $\TTT^0_Y$. Thus $\TTT^S_Y \subseteq \TTT^0_Y$ which gives $\TTT^S_Y=\TTT^0_Y$.
\end{enumerate}
\begin{proof} (\emph{\textbf{Alt}}) We know that the order topology is generated by the subbasis of all rays $(a,+\infty)$ and $(-\infty,a)$ for every $a \in X$. Suppose $Y \subseteq X$ is a convex set. Note that, the subspace topology of $Y$, $\TTT^S_Y$ is generated by the subbasis $\{(a,+\infty) \cap Y, (-\infty,a) \cap Y: a \in X\}$. Let $\TTT^0_Y$ denote the order topology of $Y$, generated by the subbasis $\{(x,+\infty)_Y, (-\infty,x): x \in Y\}$ where $(x,+\infty)_Y:=\{z \in Y: x<z\}$. 
    \\\\ \textbf{First we prove that $\TTT^0_Y \subseteq \TTT^S_Y$ (which does not require convexity)}
    \\\\ Let $U=(x,\infty)_Y=\{z \in Y: x<z\}=(x,\infty)_X \cap Y$ for $x \in Y$. This is easily seen to be a subbasis element of $\TTT^S_Y$. Likewise, it is easy to see that $(-\infty,x)_Y$ for $x \in Y$ is also a subbasis element of $\TTT^S_Y$. So any arbitrary union of finite intersection of subbases of $\TTT^0_Y$ ends up being an arbitrary union of a finite intersection of subbasis elements of $\TTT^S_Y$, which makes any set open in $\TTT^0_Y$ open in $\TTT^S_Y$. Hence $\TTT^0_Y \subseteq \TTT^S_Y$.
    \\\\ \textbf{Now we prove that $\TTT^S_Y \subseteq \TTT^0_Y$ (here convexity is needed) }
    \\\\ Let $U=(x,\infty)_X\cap Y$ for $x \in X$. If $x \in Y$, then this becomes a subbasis element. Suppose $x$ is not in $Y$. If $x$ is neither an upper bound, nor a lowerbound of $Y$, then it ends up falling into $Y$ from convexity. Therefore, it either is a lowerbound, in which case $U$ becomes $Y$ itself, or its an upper bound, in which case $U$ becomes empty. Similarly, suppose $U=(-\infty,x)_X \cap Y$ for $x \in X$. If $x \in Y$, this is precisely a subbasis element. If not, its either a lower / upper bound, making this again, a subbasis element of $\TTT^0_Y$. Hence $\TTT^S_Y \subseteq \TTT^0_Y$ which gives $\TTT^S_Y =\TTT^0_Y$.
    
    
    \end{proof}
} 

Note that, in proving that $\TTT^0_Y \subseteq \TTT^S_Y$, we do not make use of the convexity of $Y$. This motivates the following theorem:
\thmp{Let $X$ be an ordered set with the order topology $\TTT^0_X$. Let $Y$ be a subset of $X$. The subspace topology $\TTT^S_Y$ on $Y$ is always finer than (or equal to) the order topology $\TTT^0_Y$ on $Y $ inherited from the order on $X$}{The first part of the previous theorem}
\lemp{If there are two topologies on $X$ such that $\TTT \subset \TTT'$ (strict inclusion), then we have that, for any subspace $A$, the subspace topologies obey $\TTT^S_A\subseteq \TTT'^S_A$, where equality can be achieved.}{We have that $\TTT\subseteq \TTT'$. $\TTT^S_A:=\{U \cap A: U \in \TTT\}$ and $\TTT'^S_A:=\{U' \cap A: U' \in \TTT'\}$. From here we see that every set open in $\TTT^S_A$ is also open in $\TTT'^S_A$ which gives $\TTT^S_A \subseteq \TTT'^S_A$. But even if $\TTT \subsetneq \TTT'$, its possible that $\TTT^S_A=\TTT'^S_A$. Consider the discrete topology $\TTT_D=\{\varnothing, \{0\}, \{1\}, \{0,1\}\}$ on $\{0,1\}$ and consider the Sirpienski topology $\TTT_S=\{\varnothing, \{0\}, \{0,1\}\}$ on $\{0,1\}$. Let $A=\{0\}$. Then the subspace  topology of $A$ from the Discrete topology is the same as the subspace topology on $A$ from the Sirpienski topology.}
\defn{Equivalent bases ($\B \equiv \B'$)}{Two bases $\B$ and $\B'$ are said to be equivalent if $\TTT(\B)=\TTT(\B')$}
\thmp{$\B \equiv \B'$ if and only if both of the following conditions hold: \begin{enumerate}
\item For all $x \in X$ and $U \in \B$, if $x \in U$, then there exists $U' \in \B'$ so that $x \in U' \subseteq U$
\item For all $y \in X$ and $U' \in \B'$, if $y \in U'$, then there exists $U \in \B$ so that $y \in U \subseteq U'$
\end{enumerate}}{$\implies$) If $\TTT(\B)=\TTT(B')$, that means that, if we take a basis element of $\TTT(B')$, say $U'$, it is an arbitrary union of elements of $B$, and likewise any $U \in \B$ is an arbitrary union of elements of $\B'$, hence its easy to see both the properties holding.
\\\\ $\impliedby$) Suppose it is such that for every $x \in U \in \B$ we have $U'_x$ so that $x \in U'_x \subseteq U$, then $U=\cup_x U'_x$ making $U$ an arbitrary union of elements of $\B'$ giving us $\TTT(\B)\subseteq \TTT(B')$
\\\\ Likewise, it is easy to see that if the 2nd condition holds, every element of $\B'$ would be a union of elements of $\B$, making $\TTT(\B') \subseteq \TTT(\B)$. Thus $\TTT(\B)=\TTT(\B')$ implying that $\B \equiv \B'$.}
\defn{Open Map}{A map is $f:X \to Y$ (where $X$ and $Y$ are topological spaces) is said to be an \emph{\textbf{open map}} if $f(U) \in \TTT_Y$ for all $U \in \TTT_X$}
\lemp{(\textbf{Projections are open}) $\pi_1:X \times Y \to X$ and $\pi_2:X \times Y \to Y$ are open maps.}{Let $G=(U_{\alpha} \times V_{\alpha})$ be open in $\TTT_{X \times Y}$. $\pi_1(G)=\pi_1(\cup_{\alpha}(U_{\alpha}\times V_{\alpha}))=\cup_{\alpha}(\pi_1(U_{\alpha}\times V_{\alpha}))=\cup_{\alpha} U_{\alpha}$ where $U_{\alpha}$ is open in $\TTT_X$ (by definition). Likewise, $\pi_2$ is also an open map.}
\thmp{Let $Q=X=X'$ and $R=Y=Y'$, let $\TTT$ and $\TTT'$ be two topologies on $Q$ (that, respectively, contain $X$ and $X'$) and let $\U$ and $\U'$ be two topologies on $R$ that respectively contain $Y$ and $Y'$. Then we have: \begin{enumerate}
\item If $\TTT \subset \TTT' $ and $\U \subset \U'$, then $\TTT_{X \times Y} \subseteq \TTT_{X' \times Y'}$
\item If $\TTT_{X \times Y} \subseteq \TTT_{X' \times Y'}$, then $\TTT \subseteq \TTT'$ and $\U \subseteq \U'$
\end{enumerate} }{$\implies$) We are told that $\TTT \subseteq \TTT'$ and $\U \subseteq U'$, i.e, any open set in $\TTT$ is open in $\TTT'$ and any open set in $\U$ is open in $\U'$. Consider $\TTT_{X \times Y}:=\langle \{A \times B: A \in \TTT, B \in \U \}  \rangle$ and $\TTT_{X' \times Y'}:=\langle \{E \times F: E \in \TTT', F \in \U'\}\rangle$. Consider any basis element of $\TTT_{X \times Y}$, $A \times B$ for $A \in \TTT$ and $B \in \U$. From hypothesis, $A \in \TTT'$ and $B \in \U'$ implying that $A \times B$ is also a basis element of $\TTT_{X' \times Y'}$. This immediately tells us that $\TTT_{X \times Y} \subseteq \TTT_{X' \times Y'}$
\\\\ $\impliedby)$ Suppose that $\TTT_{X \times Y} \subseteq \TTT_{X' \times Y'}$. We want to show that $\TTT \subseteq \TTT'$ and $\U \subseteq \U'$. Let $A \times B$ so that $A \in \TTT$ and $B \in \U$ be any basis element of $\TTT_{X \times Y}$. We have that $A \times B= \cup_{\alpha}(E_{\alpha} \times F_{\alpha})$ for $E_{\alpha} \in \TTT_{X'}$ and $F_{\alpha} \in \U'$. Let $(x,y) \in A \times B$. This means that $x \in A$ and $y \in B$ and that there is some $\alpha$ so that $(x,y) \in E_{\alpha} \times F_{\alpha}$, or $x \in E_{\alpha} \in \TTT' $ and $y \in F_{\alpha} \in \U'$. This gives us that, for any open set $A$ in $\TTT$ and for every point $x \in A$, there exists $E_x \in \TTT'$ so that $x \in E_x \subseteq A$, which makes $A$ an arbitrary union of elements from $\TTT'$. Likewise, any open set $B \in \U$ is the arbitrary union of elements from $\U'$. That means that $\TTT \subseteq \TTT'$ and $\U \subseteq \U'$.  }
\begin{example} (\textbf{Random Example})
    $\RR^2$ generated by the product topology of $\RR$ has a basis $\{E \times F: E,F \in \TTT_R\}$ but since we know sets of the kind $(a,b)$ for $a,b$ rational forms a basis for $\RR$, and that, if $\C$ and $\D$ forms a basis for $X$ and $Y$, $\{C \times D: C\in \C, D\in \D\}$ forms a basis for $X \times Y$, meaning $\RR \times \RR$ has a basis of the kind $\{(a,b) \times (c,d):a,b,c,d \in \QQ\}$
\end{example}
\lemp{\textbf{(Useful Lemma)} If $A \subseteq X$ and $B \subseteq Y$, then $A \times B \subseteq X \times Y$ is open (in product topology) if and only if $A$ and $B$ are respectively open in $X$ and $Y$. }{
$\impliedby$) Trivial
\\\\$\implies$) If $A\times B$ is open in $X \times Y$, it is of the form $\cup_{q} (E_q \times F_q)$ for $E_q \in \TTT_X$ and $F_q \in \TTT_Y$. Let $x \in A$ and $y \in B$ so that $x,y \in A \times B \subseteq X \times Y$. There exists $q$ such that $x \in E_q$ and $y \in F_q$. This means that $A$ and $B$ each are an arbitrary union of open sets of $X$ and $Y$ respectively. We are done.}
\thmp{\textbf{(For later purposes)} If $Y$ is dense in $X$ and $X$ is an ordered set with the order topology, then $\TTT^S_Y=\TTT^0_Y$}{later}
\end{document}